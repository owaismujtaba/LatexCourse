\documentclass[12pt,a4paper]{article}
%\documentclass{report}
\usepackage[utf8]{inputenc}
\usepackage{fontenc}
\usepackage{mathptmx}
\usepackage{amsmath}
\usepackage{graphicx}
\usepackage{ifluatex}
\usepackage{xcolor}
\usepackage{framed}

\usepackage[left=2cm,right=3cm,top=2cm,bottom=2cm]{geometry}

\title{My Document}
\author{Owais Mujtaba Khanday}

\begin{document}
	\maketitle
	
	\section{Introduction}
	This document serves as an introductory guide to the core concepts and features of LaTeX. It demonstrates various text formatting techniques, including font styles, font families, text rotations, and color manipulations. By working through the examples and exercises provided, users will gain a practical understanding of how to structure and style professional documents using this powerful typesetting system.
	\section{Different Text Styles}
	These are the different Text Styles
	\textbf{Bold Text}, \textit{Italic Text}, \textsc{Captilized}
	\section{Different Font Styles}
	\fontfamily{ptm}\selectfont ptm family selected, 
	\fontfamily{qcr}\selectfont qrc family selected,
	\fontfamily{ppl}\selectfont ppl family selected,
	\fontfamily{put}\selectfont put family selected,
	
	\section{Text rotation and coloring}
	\rotatebox{10}{10 degrees}, 
	\rotatebox{20}{20 degrees},
	\colorbox{yellow}{yellow text background, \color{blue} blue text yellow background}
	
	\fbox{\colorbox{yellow}{Added fbox to higlight border, \color{blue} blue text yellow background}}
	
	\makebox[5cm][r]{makebox with width of 5cm}
	\rotatebox[origin=c,x=10pt,y=5pt,units=-360]{0}
	{	
		\colorbox{yellow}{yellow}
		\colorbox{olive}{olive}
	}
	\section{Exercise}
		Create a LaTeX document that includes:
	
		\subsection{1}
			Your name rotated 90 degrees.	
		\subsection{2} 
			A word with each letter at a different height (like a wave).	
		\subsection{3}
		 	A text box with color and border background.	
		\subsection{4} 
			A phrase where some words are rotated at different angles.	
	\section{Solution}
		These are the solutions for the exercise
		\subsection{1}
			\rotatebox[origin=, x= 20pt, y=30pt, units=-360]{90}{Owais Mujtaba Khanday}
		\subsection{2}
			\textbf{Wave Word:} 
			\raisebox{0pt}{W}%
			\raisebox{2pt}{a}%
			\raisebox{5pt}{v}%
			\raisebox{7pt}{y}%
			\raisebox{5pt}{n}%
			\raisebox{2pt}{e}%
			\raisebox{0pt}{s}%
			\raisebox{-2pt}{s}
		\subsection{3}
			\fbox{\colorbox{green}{text with green background in frame}}
		\subsection{4}
			The text below is with 10 degree increment from the origin 
			
			\rotatebox{10}{I} \rotatebox{20}{am} \rotatebox{30}{rotating} \rotatebox{40}{10 degrees} \rotatebox{50}{each} \rotatebox{60}{word}
		
	\section{Framed}
		The section shows how to put any text in a frame and choose the background color as well
		
		\begin{framed}
			This is a framed paragraph
		\end{framed}
			\definecolor{shadecolor}{rgb}{1,0.9,0.9}
		\begin{shaded}
			This is framed colors
		\end{shaded}
		This is the equation
		
		\begin{equation}
			This is the equation === 		
			\label{eq:onda}
			\frac{\partial^2 u}{\partial t^2} = c^2 \nabla^2 u
		\end{equation}


	\section{Conclusion}
	LaTeX is a powerful typesetting system widely used for production of scientific and mathematical documentation. It allows authors to focus on the content rather than the formatting, as the system handles the layout based on predefined styles and structures. This separation of content and presentation makes it an essential tool for creating professional documents, from research papers to complex books.

\end{document}