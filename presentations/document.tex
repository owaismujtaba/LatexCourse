\documentclass{beamer}
\usetheme{Madrid}
\usecolortheme{beaver} 
\usepackage[utf8]{inputenc}
\usepackage[english]{babel}
\usepackage{booktabs}
\usepackage{colortbl}
\usepackage{tcolorbox}
\usepackage{graphicx} % Required for images


\title{Fragmentation Dynamics in Global Trade}
\subtitle{G7 vs. BRICS+: Decoupling, Re-routing, and Systemic Vulnerability}
\author{Owais Mujtaba}
\date{2010--2024 Analysis}

\begin{document}

	\begin{frame}
		\titlepage
	\end{frame}
	

	\begin{frame}{Agenda}
		\tableofcontents
	\end{frame}
	

	\section{Strategic Decoupling}
	\begin{frame}{Declining G7--BRICS+ Trade Integration}
		\begin{figure}
			\centering
			\includegraphics[width=0.7\textwidth, height=0.4\textheight]{G7_BRICS_Decoupling_Figure.pdf}
			\setbeamerfont{caption}{size=\footnotesize}
			\caption{\textbf{Direct Decoupling:} Trade share ($S_t$) as \% of global commerce. Shaded area (Post-2018) shows accelerated decoupling following the trade war.}
		\end{figure}
		\small
		\begin{itemize}
			\item \textbf{Significant Downward Trend:} Regression analysis confirms a steady decline with a year coefficient of $\beta = -0.0177$ ($p < 0.001$), indicating a consistent annual erosion of direct bilateral trade.
			\item \textbf{Geopolitical Inflection:} The shaded post-2018 area highlights an acceleration in decoupling, where trade organization shifts from purely economic efficiency to geopolitical alignment.
		\end{itemize}
	\end{frame}
	
	
	
	\section{Network Structure}
	\begin{frame}{Figure 2: Absence of Structural Fragmentation}
		\begin{figure}
			\centering
			\includegraphics[width=0.7\textwidth, height=0.4\textheight]{H3_Modularity.pdf} 
			\setbeamerfont{caption}{size=\footnotesize}{
			\caption{\textbf{Inertial Integration:} Network modularity ($Q$) remains negative, indicating cross-bloc dependencies outweigh regional clustering.}}
		\end{figure}
		
		{\tiny
			\begin{itemize}
				\item \textbf{Rejection of Fragmentation (H3):} Mann-Kendall tests ($p = 0.075$) confirm the absence of a fragmentation trend. Modularity ($Q$) stays below zero, signifying that nodes connect across group peripheries rather than within isolated blocs.
				\item \textbf{Resilience via Interdependence:} During the 2020 pandemic, modularity reached its most negative point ($Q = -0.031$), suggesting crises actually escalate the functional necessity of cross-bloc partnerships over "friend-shoring."
			\end{itemize}
		}
	\end{frame}
	

	\section{Geopolitical Alignment}
	\begin{frame}{Geopolitical Realignment (ARI)}
		\begin{figure}
			\centering
			\includegraphics[width=0.7\textwidth, height=0.4\textheight]{trade_alignment_ari_zscore}
			\caption{\textbf{Regime Shift:} The Adjusted Rand Index (ARI) spike in 2024 shows trade clusters finally aligning with geopolitical blocs ($p < 0.01$).}
		\end{figure}
	\end{frame}
	
	% --- SECTION 4: Systemic Vulnerability ---
	\section{Systemic Vulnerability}
	\begin{frame}{Figure 4: Resilience Loss Index (RLI)}
		\begin{figure}
			\centering
			\includegraphics[width=0.7\textwidth, height=0.4\textheight]{RLI_ZScore}
			\caption{\textbf{Systemic Fragility:} The rise in RLI through 2023 indicates trade concentration in critical "bottleneck" states.}
		\end{figure}
	\end{frame}
	
	% --- SECTION 5: Summary ---
	\section{Conclusion}
	\begin{frame}{Summary of Findings}
		\begin{itemize}
			\item \textbf{Hypothesis 1 (Confirmed):} Significant decline in G7-BRICS+ trade share.
			\item \textbf{Hypothesis 3 (Confirmed):} Global trade network remains topographically unified.
			\item \textbf{Final Insight:} We are seeing a "structural rupture"—decoupling in volume, but not in network connectivity.
		\end{itemize}
		
		\begin{tcolorbox}[colback=red!5, colframe=red!75!black, title=Future Outlook]
			The 2024 ARI spike suggests the beginning of a "re-binarization" of the international economic order.
		\end{tcolorbox}
	\end{frame}
	
	% Final Slide
	\begin{frame}
		\centering
		\Huge Questions?
		\vspace{1cm}
		\normalsize \textbf{Thank you for your attention!}
	\end{frame}
	
\end{document}