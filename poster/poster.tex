\documentclass[25pt, a0paper, portrait]{tikzposter}
\usepackage[utf8]{inputenc}
\usepackage{graphicx}
\usepackage{caption}
\usepackage{booktabs}
\usepackage{xcolor}
\usepackage{amsmath}
\usepackage{amsfonts}
\usepackage{bm}
\usepackage{cmbright}


\title{\parbox{\linewidth}{\centering \textbf{The Architecture of Alignment:} \\ \huge Quantifying Geoeconomic Fragmentation}}
\author{Owais Mujtaba}
\institute{CITIC | University of Granada}


\definecolor{primary}{HTML}{1E1B4B}  
\definecolor{accent}{HTML}{059669}  
\definecolor{bgLight}{HTML}{F8FAFC}  
\definecolor{blockTitle}{HTML}{F1F5F9}

\definecolorstyle{AcademicModern}{
	\definecolor{colorOne}{HTML}{F8FAFC}   
	\definecolor{colorTwo}{HTML}{E2E8F0}   
	\definecolor{colorThree}{HTML}{1E1B4B} 
	\colorlet{backgroundcolor}{colorOne}
	\colorlet{blocktitlebgcolor}{colorTwo}
	\colorlet{blocktitlefgcolor}{primary}
	\colorlet{blockbodybgcolor}{white}
	\colorlet{blockbodyfgcolor}{black}}

\newcommand{\boldemeq}[1]{{\color{accent}\bm{#1}}}

\usetheme{Board}
\usecolorstyle{AcademicModern}

\begin{document}
	\maketitle
	
	\begin{columns}
		% --- COLUMN 1: Introduction & Hypotheses ---
		\column{0.25}
			\block{Introduction}{				
				\begin{itemize}
					\item \textbf{The Post-Cold War Paradigm:} The era of hyperglobalization was defined by market efficiency superseding Westphalian security concerns \cite{friedman2005}.
					
					\item \textbf{Systemic Rupture:} The liberal consensus is destabilized by:
					\begin{itemize}
						\item \textbf{U.S.--China trade war:} Strategic decoupling.
						\item \textbf{COVID-19 pandemic:} Supply chain brittleness.
						\item \textbf{Invasion of Ukraine:} Weaponized interdependence.
					\end{itemize}
					
					\item \textbf{Research Question:} Is the global system structurally bifurcating into rival blocs or merely adapting routes within a singular integrated topology?
					
					\item \textbf{Approach:} Moving beyond gravity models toward \textbf{Network Topology Analysis} to identify shifts from global integration to modular intra-bloc structures.
				\end{itemize}
			}
			
			\block{Research Hypotheses}{
			\textbf{H1: G7--BRICS Strategic Decoupling} \\
			Quantifying the trade relationship using bilateral export data:
			\begin{center}
				\[ \boldemeq{ S_t = \frac{\sum_{i \in \mathcal{G}} \sum_{j \in \mathcal{B}} X_{i,j,t} + \sum_{i \in \mathcal{B}} \sum_{j \in \mathcal{G}} X_{i,j,t}}{\sum_{i} \sum_{j} X_{i,j,t}} } \]
			\end{center}
			
			\textbf{H2: Consensus Connector Stability} \\
			Structural influence measured through weighted betweenness centrality:
			\begin{center}
				\[ \boldemeq{ S_i = 1 - \frac{\sigma_{BC}}{\mu_{BC}} } \]
			\end{center}
			
			\textbf{H3: Network Modularity} \\
			Newman-Girvan modularity index to detect systemic fragmentation:
			\begin{center}
				\[ \boldemeq{ Q_t = \frac{1}{2m_t} \sum_{i,j} \left( A_{ij}^t - \frac{k_i^t k_j^t}{2m_t} \right) \delta(c_i, c_j) } \]
			\end{center}
			
			\textbf{H4: Community Alignment} \\
			Validation via Z-scores derived from Monte Carlo permutations:
			\begin{center}
				\[ \boldemeq{ Z = \frac{ARI_{obs} - \mu_{null}}{\sigma_{null}} } \]
			\end{center}
			
			\textbf{H5: Systemic Fragility} \\
			Resilience Loss Index (RLI) based on global network efficiency:
			\begin{center}
				\[ \boldemeq{ \text{RLI}_t = 1 - \frac{E_t(\mathcal{G}_t \setminus \mathcal{C}_t)}{E_t(\mathcal{G}_t)} } \]
			\end{center}
		}
		
		% --- COLUMN 2: Visualizations & Conclusion ---
		\column{0.75}
			\block{The Decline of G7--BRICS+ Trade Interdependence}{
				\begin{center}
					\includegraphics[width=0.9\linewidth]{G7_BRICS_Decoupling_Figure.pdf}
					\captionof{figure}{\textbf{Declining G7--BRICS+ Trade Integration (2010--2024).} 
						Panel (a) shows trade share $S_t$; Panel (b) shows directional flows. 
						Shaded area highlights post-2018 decoupling.}
					\label{fig:G7_BRICS}
				\end{center}
				
				\begin{itemize}
					\item \textbf{Significant Trade Contraction:} The G7--BRICS+ share of global trade ($S_t$) fell from 6.1\% (2010) to 4.8\% (2024), marking a 21\% reduction in inter-bloc integration.
					\item \textbf{Negative Growth Trend:} Log-linear regression identifies an annual proportional decline of $\approx$1.8\% ($\beta = -0.0177, p < 0.001$), indicating a steady erosion of economic ties.
					\item \textbf{Post-2018 Structural Break:} Analysis reveals a significant hiatus coinciding with U.S.-China tensions; the post-2018 average trade share dropped by $\Delta S = -0.72$ percentage points ($p = 0.0006$).
					\item \textbf{Decoupling Toll:} The 2024 trade share is 1.67\% lower than levels projected by pre-2018 trajectories, quantifying the cumulative impact of geopolitical decoupling.
					\item \textbf{Increased Volatility:} Post-2018 trade flow variance increased by a factor of 3.60, reflecting the growing influence of non-economic factors like sanctions and supply chain security.
					\item \textbf{Robustness Testing:} Placebo testing (2014) showed minimal impact, confirming that the structural shift is uniquely linked to post-2018 geopolitical fragmentation.
				\end{itemize}
			}
			
			\block{The Persistence of Strategic Trade Intermediaries}{
				\begin{itemize}
					\item \textbf{The ``Stability Triad'':} Network centrality analysis identifies the USA, China, and France as the dominant global brokers, consistently maintaining mean betweenness scores $> 0.025$.
					\item \textbf{US Structural Invariance:} The USA remains the primary coordinate for transatlantic and transpacific trade with the highest stability index ($S_i = 0.92$).
					\item \textbf{European Anchors:} France ($S_i = 0.77$) has emerged as a stability anchor, slightly surpassing Germany ($S_i = 0.74$) due to its institutional role in the EU and Francophone networks.
					\item \textbf{Logistical Gatekeepers:} Mid-tier economies like the Netherlands ($S_i = 0.80$) and Spain ($S_i = 0.83$) function as ``stabilizing bridges,'' where port and financial infrastructure protect nodes from geopolitical turmoil.
					\item \textbf{Structural Endpoint Paradox:} Emerging manufacturing hubs like Vietnam ($S_i = -0.66$) and Mexico ($S_i = 0.33$) exhibit near-zero brokerage centrality ($BC < 0.0004$), acting as production endpoints rather than systemic intermediaries.
					\item \textbf{Systemic Influence vs. Volume:} Results indicate that high export volumes do not automatically translate into structural power; systemic influence depends on a state's role as a bridge between disconnected partners.
				\end{itemize}
			
			}
			
			\block{The Absence of Network Fragmentation}{
				\begin{center}
					\includegraphics[width=\linewidth]{H3_Modularity.pdf}
					\caption{\textbf{Absence of Structural Fragmentation, 2010--2024.} 
						Negative modularity ($Q$) indicates that G7 and BRICS+ economies remain structurally connected across blocs despite geopolitical tensions.}
					\label{fig:H3}
				\end{center}
				\begin{itemize}
					\item \textbf{Rejection of Fragmentation (H3):} Despite declining bilateral trade, the global system has not split into discrete regional blocs. Mann-Kendall tests confirm no significant increase in modularity ($p = 0.075$).
					\item \textbf{Persistent Negative Modularity:} The network maintains a "disassortative" arrangement ($-0.015 > Q > -0.031$), indicating that G7 and BRICS+ economies remain structurally inseparable and continue to trade across bloc boundaries.
					\item \textbf{Functional Complementarity:} Deep-seated dependencies—such as German reliance on Chinese rare earths and U.S. need for Indian pharmaceuticals—create manufacturing ties that political will alone cannot easily reconfigure.
					\item \textbf{Crisis-Induced Integration:} During the COVID-19 pandemic, modularity reached its most negative value ($Q = -0.031$), suggesting that supply chain stress actually forces states to leverage existing cross-bloc interdependencies.
					\item \textbf{Inertial Integration:} Geopolitical shocks (2018 Trade War, 2022 Ukraine Invasion) failed to trigger positive modularity, revealing a topological core that remains unified despite "de-risking" rhetoric.
					\item \textbf{Prohibitive Decoupling Costs:} The absence of fragmentation suggests that the logistical and economic costs of restructuring core supply chain networks are currently prohibitive for major powers.
				\end{itemize}
			}
			
			\block{The Emergence of Geopolitical Bloc Formation}{
				\begin{center}
					\includegraphics[width=0.80\linewidth]{trade_alignment_ari_zscore.pdf}
					\caption{\textbf{Geopolitical Realignment of Global Trade, 2010--2024.} }
				\end{center}
				\begin{itemize}
					\item \textbf{2024 Regime Shift:} Community detection algorithms reveal a fundamental shift from ``bloc-blind'' trade to alignment based on geopolitical orientation ($Z\text{-score} = 3.06, \, p < 0.01$).
					
					\item \textbf{Historical Baseline (2010--2023):} Trade was previously driven by economic complementarity and profit rather than strategy, with low correlation between trade clusters and political blocs ($\text{ARI} < 0.15$).
					
					\item \textbf{Rapid Re-binarization:} In 2024, the alignment index leaped to $\text{ARI} = 0.28$, indicating that states are now significantly more likely to trade with ``like-minded'' partners within their own bloc.
					
					\item \textbf{Drivers of Realignment:} This crystallization is fueled by the 2023 BRICS+ expansion and G7 ``de-risking'' policies (e.g., CHIPS Act, Critical Raw Materials Act), mirroring a Cold War-style bipolar topology.
					
					\item \textbf{The ARI-Modularity Paradox:} While the system remains topologically unified (negative modularity), bilateral flows are increasingly concentrated within blocs (rising ARI), marking a transition from hyperglobalization to strategic alignment.
					
					\item \textbf{End of an Era:} These trends suggest the conclusion of the post-1991 era, replaced by a system where logistical connectivity persists despite a heavy geopolitical overlay on trade choices.
				\end{itemize}
			}
			
			\block{The Evolution of Systemic Vulnerability}{
				\begin{center}
					\includegraphics[width=\linewidth]{RLI_ZScore.pdf}
					\caption{\textbf{Temporal Evolution of Systemic Vulnerability, 2010--2024.} 
						RLI measures network interruption via connector removal; higher values denote fragility. 
						The 2023 peak highlights reliance on bottlenecks, while the 2024 dip reflects diversification efforts.}
					\label{fig:RLI}
				\end{center}
				\begin{itemize}
					\item \textbf{Systemic Fragility (H5):} Analysis confirms a 15-year trajectory of increasing vulnerability, followed by a recent shift toward restructuring and diversification.
					\item \textbf{The Efficiency-Resilience Trade-off:} From 2010 to 2023, the Resilience Loss Index (RLI) rose by 18.5\% (0.081 to 0.096), reflecting a global system increasingly dependent on a restricted set of structural bottlenecks.
					\item \textbf{Diffuse Vulnerability:} Non-significant Z-scores (between -1.0 and 1.0) suggest that fragility is not limited to "super-connectors" but is a diffuse characteristic across a broad class of neutral intermediary nodes.
					\item \textbf{The 2024 Inflection Point:} The RLI saw its first significant decrease in a decade (falling to 0.089), signaling an incipient move toward supply chain diversification and "China Plus One" strategies.
					\item \textbf{Systemic Recoupling:} Recent implementation of industrial policies and export controls indicates that state intervention is beginning to counterbalance market-driven concentration, prioritizing security over pure efficiency.
					\item \textbf{Future Outlook:} The 2024 downturn raises a critical question: is this the start of a permanent shift toward a more resilient, albeit more expensive, global trading framework?
				\end{itemize}
			}
			
		\block{Conclusion}{
			\begin{itemize}
				\item \textbf{Inertial Integration:} Despite "weaponized" interdependence, the trade network shows high resistance to modular fragmentation. Persistent negative modularity reflects functional complementarities in minerals, semiconductors, and pharmaceuticals that exceed current political will for decoupling.
				\item \textbf{Structural vs. Material Power:} Systemic influence remains rooted in institutional capacity and logistical geography. While Vietnam and Mexico have grown as manufacturing hubs, they remain "structural endpoints" rather than autonomous intermediaries like the \textit{Stability Triad} (USA, China, France).
				\item \textbf{2024 Structural Rupture:} Data signals the end of "bloc-blind" commerce. A surge in the Adjusted Rand Index (ARI) alongside the first decade-long dip in the Resilience Loss Index (RLI) suggests a fundamental rewiring of the global trade topology.
				\item \textbf{Strategic Trade-off:} The global order is shifting from the efficiency-led "hyperglobalization" of the post-1991 era to a "re-binarized" system that prioritizes security and resilience, albeit at higher transaction costs.
				\item \textbf{Role of Bridge Nations:} The future stability of the international order depends on "stabilizing bridges" (e.g., Netherlands, Spain) to maintain the attenuating threads of cross-bloc connectivity in a high-friction, bipolar world.
			\end{itemize}
		}
		\end{columns}
\end{document}