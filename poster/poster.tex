\documentclass[final]{beamer}

\usepackage[size=a0,scale=1.4]{beamerposter} % Standard A0 size
\usetheme{Madrid} % Base theme for structure
\usecolortheme{beaver}

% --- Packages ---
\usepackage{multicol}
\usepackage{booktabs}
\usepackage{tcolorbox}
\usepackage{graphicx}
\usepackage{colortbl}
\usepackage{caption}

% --- Customizing Colors ---
\setbeamercolor{block title}{fg=white,bg=red!75!black}
\setbeamercolor{block body}{fg=black,bg=gray!10}

\title[Global Trade Dynamics]{Fragmentation Dynamics in Global Trade: G7 vs. BRICS+}
\author{LaTeX Advanced Course Project}
\institute{Network Science \& Geopolitics Department}

\begin{document}
	\begin{frame}[t] % [t] aligns content to the top
		
		% --- Header Region ---
		\begin{tcolorbox}[colback=red!75!black, colframe=black, arc=0mm]
			\centering
			\color{white}
			\Huge \textbf{Fragmentation Dynamics in Global Trade} \\
			\Large \textit{G7 vs. BRICS+: Decoupling, Re-routing, and Systemic Vulnerability (2010--2024)}
		\end{tcolorbox}
		
		\vspace{1cm}
		
		\begin{columns}[t,onlytextwidth] % Three-column layout, use full width
			
			% --- COLUMN 1 ---
			\begin{column}{.32\textwidth}
				
				\begin{block}{1. Introduction: Rupture of the "Flat World"}
					The post-Cold War "flat world" paradigm of seamless global markets is fracturing. Recent systemic crises—the US-China trade war, COVID-19, and the Ukraine invasion—have replaced market efficiency with a "logic of security" and dirigiste industrial policy.
				\end{block}
				\vfill
				\begin{block}{2. The Five Hypotheses}
					\begin{itemize}
						\item \textbf{H1 Strategic Decoupling:} Declining G7--BRICS+ bilateral trade.
						\item \textbf{H2 Connector Stability:} Invariant structural hub positions.
						\item \textbf{H3 Bloc Fragmentation:} Multi-polar modularity ($Q$).
						\item \textbf{H4 Geopolitical Alignment:} Clusters aligning with blocs.
						\item \textbf{H5 Systemic Fragility:} Vulnerability via bottlenecks.
					\end{itemize}
				\end{block}
				\vfill
				\begin{block}{3. Data \& Methodology}
					Analyzing WITS/UN Comtrade flows (2010--2024):
					\begin{itemize}
						\item \textbf{Network Modularity ($Q$)} \& \textbf{Leiden Algorithm} for clustering.
						\item \textbf{Adjusted Rand Index (ARI)} for geopolitical alignment.
						\item \textbf{Resilience Loss Index (RLI)} via global efficiency $E_t$.
					\end{itemize}
				\end{block}
				\vfill
			\end{column}
			
			% --- COLUMN 2 ---
			\begin{column}{.32\textwidth}
				
				\begin{block}{4. Visual Analysis: Decoupling vs. Connectivity}
					\centering
					\includegraphics[width=0.9\textwidth]{G7_BRICS_Decoupling_Figure.pdf}
					\captionof{figure}{\textbf{Direct Decoupling:} G7--BRICS+ trade share contracted 21\% since 2010. Annual proportional decline of $\sim$1.8\% ($\beta = -0.0177, p<0.001$).}
					
					\vspace{0.8cm}
					
					\includegraphics[width=0.9\textwidth]{H3_Modularity.pdf}
					\captionof{figure}{\textbf{Inertial Integration:} $Q$ remains negative ($-0.015 > Q > -0.031$), proving G7 and BRICS+ are topologically inseparable despite decoupling.}
				\end{block}
				\vfill
				\begin{tcolorbox}[colback=blue!5, colframe=blue!75!black, title=The Stability Triad]
					USA ($S_i=0.92$), China ($S_i=0.78$), and France ($S_i=0.77$) maintain invariant structural hub positions. Logistical path dependency protects these nodes from political turmoil.
				\end{tcolorbox}
				\vfill
			\end{column}
			
			% --- COLUMN 3 ---
			\begin{column}{.32\textwidth}
				
				\begin{block}{5. "Inertial Integration"}
					The global economy exhibits a state where the topological core remains unified even as bilateral ties fray.
					\begin{itemize}
						\item \textbf{Prohibitive Costs:} Reconfiguring deeply embedded manufacturing networks is too expensive.
						\item \textbf{Functional Necessity:} Cross-bloc dependence on rare earths, semiconductors, and APIs.
					\end{itemize}
				\end{block}
				\vfill
				\begin{block}{6. Realignment \& Resilience Loss}
					The \textbf{Resilience Loss Index (RLI)} and \textbf{ARI} reveal:
					\begin{itemize}
						\item \textbf{2024 Realignment:} ARI spiked to 0.28 ($Z=3.06, p<0.01$), signaling "re-binarization" along geopolitical lines.
						\item \textbf{Resilience Correction:} RLI dipped in 2024 (0.089) as states diversified sourcing to reduce bottleneck dependency.
					\end{itemize}
					\centering
					\includegraphics[width=0.85\textwidth]{RLI_ZScore.pdf}
				\end{block}
				\vfill
				\begin{block}{7. Conclusions}
					\begin{enumerate}
						\item \textbf{"Inertial Integration":} Functional necessity of cross-bloc ties prevents modular fragmentation ($Q < 0$).
						\item \textbf{Structural Power:} Invariant hubs (G7/BRICS+) remain resilient to manufacturing redistribution.
						\item \textbf{The 2024 Inflection:} Unprecedented geopolitical alignment (ARI) signals the collapse of the "flat world" and a shift toward a less efficient, but potentially more resilient, bipolared order.
					\end{enumerate}
				\end{block}
				\vfill
				\begin{tcolorbox}[colback=gray!5, title=References]
					\tiny
					[1] WITS World Bank Trade Data (2024). \\ \relax
					[2] Farrell \& Newman (2019) "Weaponized Interdependence". \\ \relax
					[3] Friedman (2005) "The World is Flat". \\ \relax
					[4] Baldwin (2016) "The Great Convergence".
				\end{tcolorbox}
				\vfill
			\end{column}
		\end{columns}
		
	\end{frame}
\end{document}